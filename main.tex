\documentclass[onecolumn,12pt]{IEEEtran}
\usepackage[utf8]{inputenc}
\usepackage[left=1in,right=1in,bottom=0.95in,nohead,nofoot]{geometry}
\usepackage{graphicx}

\begin{document}
\title{Informe Laboratorio 5}
\author{Redes de Datos}
%\vspace{3mm}

\begin{figure}[h]
\includegraphics[width=0.50\textwidth]{logo_udp.png}
\label{fig:mesh1}
\\
\\
\\
\\
\\
\maketitle
\end{figure}
\begin{center}
Integrantes:\\
\hfill \\
Gonzalo Felipe\\
Andres Hernandez\\
Franco Centeno\\
\hfill \\
\hfill \\
\hfill \\
\hfill \\
\ \hfill \\
Profesor:\\
Jose Perez\\ \hfill \\
Ayudante:\\
Alexis Inzunza\\
\end{center}

\newpage
\title{Indice}
\author{ }
\maketitle
\hrule
\tableofcontents

\newpage
\section{INTRODUCCION}
\hfill \\
En este laboratorio trabajamos con el enrutamiento, el cual es el proceso para encontrar el camino mas óptimo para que un paquete llegue a su destino. Este procedimiento es importante en la comunicacion entre redes, puede haber más de un camino y el router debe ser capaz de elegir uno indicado. Hay dos maneras de enrutamiento:\\ \\
1. Enrutamiento Estático: Se llenan manualmente las rutas a seguir. Este proceso no es factible en una red muy grande, ya que dichas rutas introducirse en cada equipo, uno por uno. Hay mayor conocimiento de origen de posibles fallas ya que el administrador de la red debe interactuar con cada equipo.\\ \\
2. Enrutamiento Dinámico: Se llenan las tablas de forma automática, gracias a un algoritmo. Los protocolos de enrutamiento
dinámico se pueden clasificar como; basados en vector distancia y basados en estado de enlace. Menor manejo de red.

\section{ACTIVIDAD}
\hfill \\

La actividad consiste en asignar una IP manualmente a cada dispositivo. Se debe usar las configuraciones de ruteo tanto estático como dinámico. Las configuraciones de ruteo dinámico y estático se realizan con comandos del programa packet tracer.


\begin{figure}[h]
\begin{center}
\includegraphics[width=0.75\textwidth]{redes.png}
\label{fig:mesh1}
\end{center}
\end{figure}


\section{CONCLUSION}
\hfill \\
Gracias a los tipos de enrutamiento se puede definir las rutas con las cuales podemos interactuar entre redes, lo cual es importante dentro de cualquier ambito donde se cuente con el uso de varios equipos a la vez


\hfill \\
\hfill \\
\section{BIBLIOGRAFIA}

Packet Tracer,
\emph{Cisco} \\
\url{https://packet-tracer.programas-gratis.net/} \\


\end{document} 
